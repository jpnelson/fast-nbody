\documentclass[11pt,a4paper,oneside]{article}
\usepackage[margin=2.0cm]{geometry}
\usepackage[english]{babel}
\usepackage[normalem]{ulem}
\usepackage{amsfonts}
\usepackage{amsmath}
\usepackage{amssymb}
\usepackage{tabularx}
\usepackage{color}
\usepackage{tabto}
\newcommand{\hilight}[1]{\colorbox{yellow}{#1}}
\newcommand{\hs}{$\hspace{0.5cm}$}
\newcommand{\length}{\ell}
\newcommand{\tabseteq}{$\=$ =}
\newcommand{\tabeq}{$\>$ =}
\newcommand{\bt}{\begin{tabbing}}
\newcommand{\et}{\end{tabbing}}
\newcommand{\st}{$\hspace{0.25cm} s.t \hspace{0.25cm}$}
\begin{document}
\title{COMP3006 Research project plan}
\author{Joshua Nelson u4850020}
\maketitle
\section*{Project goal}
To better understand the Fast multipole method and the Particle mesh ewald method, and to examine which algorithm is preferable in which circumstance. Additionally, the problem of implementing these algorithms in a high level language like Java will be examined.
\section*{Research plan by week}
\begin{itemize}
\item Week 1
    \begin{itemize}
    \item Find a supervisor and topic
    \item Research topic
    \end{itemize}
\item Week 2
    \begin{itemize}
    \item Read and understand the Rokhlin-Greengard paper on the Fast Multipole method
    \item Read other papers on the Particle Mesh Ewald method.
    \end{itemize}
\item Week 3
    \begin{itemize}
    \item Write and practice presentation on my topic
    \item Start writing Report
    \end{itemize}
\item Week 4
    \begin{itemize}
    \item Give presentation
    \item Write report introduction
    \end{itemize}
\item Week 5
    \begin{itemize}
    \item Organise plan for the remaining time
    \end{itemize}
\item Week 6
    \begin{itemize}
    \item Framework / GUI created
    \item Basic algorithm implemented
    \end{itemize}
\item Week 7
    \begin{itemize}
    \item Begin implementation of the Particle mesh-ewald method
    \end{itemize}
\item Break
    \begin{itemize}
    \item Finish implementation of the Particle mesh-ewald method
    \item Implement the Fast multipole method
    \item Bug test and verify the two method's
    \end{itemize}
\item Week 8
    \begin{itemize}
    \item Improve the implementation's efficiencies
    \item Analyse the efficiencies (Include in report)
    \end{itemize}
\item Week 9
    \begin{itemize}
    \item Prepare report
    \item Prepare final presentation
    \end{itemize}
\item Week 10
    \begin{itemize}
    \item Finish preparing report
    \end{itemize}
\item Week 10
    \begin{itemize}
    \item Finish preparing presentation
    \end{itemize}
\item Week 11
    \begin{itemize}
    \item Give presentation
    \end{itemize}
\item Week 12
    \begin{itemize}
    \item Submit report
    \end{itemize}
\end{itemize}
\section*{Deliverables}
\begin{itemize}
\item Artefact: a piece of software, written in java, that implements the Fast multipole method and the Particle Mesh Ewald method for the n-body problem.
\item Report: a report detailing the comparison between these two algorithm's running times, and the interesting parts of their implementations
\item Presentation: a presentation will be given on the contents of my report and the creation of the artefact.
\end{itemize}
\end{document}
