\documentclass[10pt]{beamer}
% Class options include: notes, notesonly, handout, trans,
%                        hidesubsections, shadesubsections,
%                        inrow, blue, red, grey, brown

% Theme for beamer presentation.
\usepackage{beamerthemesplit} 
%\usepackage{times}
\usepackage{anysize}
\usepackage{fancyhdr}
\usepackage{graphicx}
\usepackage{pdfpages}
\usepackage{amsmath}
\usepackage{amssymb}
% Other themes include: beamerthemebars, beamerthemelined, 
%                       beamerthemetree, beamerthemetreebars  

\title{The Fast Multipole Algorithm vs. The Particle Mesh Ewald Method}    % Enter your title between curly braces
\author{Joshua Nelson}                 % Enter your name between curly braces
\institute{COMP3006 Research Project}      % Enter your institute name between curly braces
\date{\today}                    % Enter the date or \today between curly braces

\usetheme{PaloAlto}
\usecolortheme{beaver}

\newcommand{\bcen}{\begin{center}}
\newcommand{\ecen}{\end{center}}

\begin{document}

% Creates title page of slide show using above information
\begin{frame}
  \titlepage
\end{frame}
\note{} % Add notes to yourself that will be displayed when
        % typeset with the notes or notesonly class options



\section{The N-body problem}

\begin{frame}
    \frametitle{The N-body problem}
    \begin{itemize}
        \item<1-> The problem
        \begin{itemize}
            \item<1-> N bodies in space - calculate some interaction between them
        \end{itemize}
    \end{itemize}
\end{frame}

\begin{frame}
    \frametitle{N-body diagram}
    \bcen \includegraphics<1>[width=2.5cm,bb=60 70 150 150]{nbodies.pdf} \ecen
\end{frame}

\section{Motivation}
\begin{frame}
    Why is this useful?
\end{frame}

\begin{frame}
    \frametitle{The millenium run}
    \bcen \includegraphics<1>[width=7.5cm,bb= 0 0 2048 1536,clip]{millenium_run.jpg} \ecen
\end{frame}
\begin{frame}
    \frametitle{Plasma physics simulation}
    \bcen \includegraphics<1>[width=7.5cm,bb= 0 0 1589 1609,clip]{plasma.jpg} \ecen
\end{frame}
\begin{frame}
    \frametitle{Molecular dynamics}
    \bcen \includegraphics<1>[width=7.5cm,bb= 0 0 1144 936,clip]{molecule.jpg} \ecen
\end{frame}

\begin{frame}
    \frametitle{The basic solution}
    \begin{itemize}
        \item<1-> The na\"{\i}ve solution
        \begin{itemize}
            \item<1-> Calculate interactions between every pair of bodies
            \item<2-> $O(n^2)$ complexity for N bodies.
        \end{itemize}
    \end{itemize}
\end{frame}

\begin{frame}
    \frametitle{N-body diagram}
    \bcen \includegraphics<1>[width=2.5cm,bb=60 70 150 150]{nbodies.pdf} \ecen
    \bcen \includegraphics<2>[width=2.5cm,bb=60 0 150 150]{nbodies_arrows.pdf} \ecen
\end{frame}

\section{Alternative solutions}
\begin{frame}
    \frametitle{Alternative solutions}
    \begin{itemize}
        \item<1-> Can this be done faster?
        \begin{itemize}
            \item<2-> The fast multipole method
            \item<3-> The particle mesh ewald method
        \end{itemize}
    \end{itemize}
\end{frame}

\begin{frame}
    \frametitle{The fast multipole method}
    \begin{itemize}
        \item<1-> Form a grid and group particles in the grid
        \item<2-> Treat far away groups as singular entities, forming a function for their potentials
        \item<3-> Sum these functions
        \item<4 -> Turns out to be $O(n)$
    \end{itemize}
\end{frame}

\begin{frame}
    \frametitle{The particle mesh ewald method}
    \begin{itemize}
        \item<1-> We take the potential function, and apply a fast fourier transform over a discrete mesh, then interpolate
        \item<2-> Turns out to be $O(n \text{log} n )$
    \end{itemize}
\end{frame}

\begin{frame}
    \frametitle{Comparing}
    \begin{itemize}
        \item<1-> Which is more commonly used?
        \item<2-> Particle Mesh Ewald Method, $O(n \text{log} n )$...
    \end{itemize}
\end{frame}

\section{My project}
\begin{frame}
    \frametitle{The scope}
    \begin{itemize}
        \item<1-> Compare these two algorithms
        \item<2-> Attempt to improve the algorithms and their implementations
        \item<3-> Determine the point at which each algorithm is preferable
    \end{itemize}
\end{frame}
\end{document}

